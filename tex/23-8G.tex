2. Quia ipse super mária fundávit \textbf{e}um: \ast\  et super flúmina præpa\textit{rá}\textit{vit} \textbf{e}um.\

3. Quis ascéndet in montem \textbf{Dó}mini? \ast\  aut quis stabit in loco \textit{sanc}\textit{to} \textbf{e}jus?\

4. Innocens mánibus et mundo corde, \dag\  qui non accépit in vano ánimam \textbf{su}am, \ast\  nec jurávit in dolo pró\textit{xi}\textit{mo} \textbf{su}o.\

5. Hic accípiet benedictiónem a \textbf{Dó}mino: \ast\  et misericórdiam a Deo, salu\textit{tá}\textit{ri} \textbf{su}o.\

6. Hæc est generátio quæréntium \textbf{e}um, \ast\  quæréntium fáciem \textit{De}\textit{i} \textbf{Ja}cob.\

7. Attóllite portas, príncipes, vestras, \dag\  et elevámini, portæ æter\textbf{ná}les: \ast\  et introí\textit{bit} \textit{Rex} \textbf{gló}riæ.\

8. Quis est iste Rex glóriæ? \dag\  Dóminus fortis et \textbf{pot}ens: \ast\  Dóminus pot\textit{ens} \textit{in} \textbf{prǽ}lio.\

9. Attóllite portas, príncipes, vestras, \dag\  et elevámini, portæ æter\textbf{ná}les: \ast\  et introí\textit{bit} \textit{Rex} \textbf{gló}riæ.\

10. Quis est iste Rex \textbf{gló}riæ? \ast\  Dóminus virtútum ipse \textit{est} \textit{Rex} \textbf{gló}riæ.\

11. Glória Patri, et \textbf{Fí}lio, \ast\  et Spirí\textit{tu}\textit{i} \textbf{Sanc}to.\

12. Sicut erat in princípio, et nunc, et \textbf{sem}per, \ast\  et in sǽcula sæcu\textit{ló}\textit{rum}. \textbf{A}men.\

