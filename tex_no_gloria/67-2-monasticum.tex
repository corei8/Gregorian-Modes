2. Sicut déficit fumus, de\textbf{fí}ciant: \ast\  sicut fluit cera a fácie ignis, sic péreant peccatóres a fáci\textit{e} \textbf{De}i.\

3. Et justi epuléntur, et exsúltent in conspéctu \textbf{De}i: \ast\  et delecténtur in \textit{læ}\textbf{tí}\textbf{ti}a.\

4. Cantáte Deo, psalmum dícite nómini \textbf{e}jus: \ast\  iter fácite ei, qui ascéndit super occásum: Dóminus no\textit{men} \textbf{il}li.\

5. Exsultáte in conspéctu \textbf{e}jus: \ast\  turbabúntur a fácie ejus, patris orphanórum et júdicis vi\textit{du}\textbf{á}rum.\

6. Deus in loco sancto \textbf{su}o: \ast\  Deus, qui inhabitáre facit uníus moris \textit{in} \textbf{do}mo:\

7. Qui edúcit vinctos in forti\textbf{tú}dine, \ast\  simíliter eos qui exásperant, qui hábitant in \textit{se}\textbf{púl}cris.\

8. Deus, cum egrederéris in conspéctu pópuli \textbf{tu}i, \ast\  cum pertransíres in \textit{de}\textbf{sér}to:\

9. Terra mota est, étenim cæli distillavérunt a fácie Dei \textbf{Sí}nai, \ast\  a fácie De\textit{i} \textbf{Is}\textbf{ra}ël.\

10. Plúviam voluntáriam segregábis, Deus, hereditáti \textbf{tu}æ: \ast\  et infirmáta est, tu vero perfecís\textit{ti} \textbf{e}am.\

11. Animália tua habitábunt in \textbf{e}a: \ast\  parásti in dulcédine tua páupe\textit{ri}, \textbf{De}us.\

12. Dóminus dabit verbum evangeli\textbf{zán}tibus, \ast\  virtú\textit{te} \textbf{mul}ta.\

13. Rex virtútum dilécti di\textbf{léc}ti: \ast\  et speciéi domus divíde\textit{re} \textbf{spó}\textbf{li}a.\

14. Si dormiátis inter médios cleros, pennæ colúmbæ deargen\textbf{tá}tæ, \ast\  et posterióra dorsi ejus in palló\textit{re} \textbf{au}ri.\

15. Dum discérnit cæléstis reges super eam, nive dealbabúntur in \textbf{Sel}mon: \ast\  mons Dei, \textit{mons} \textbf{pin}guis.\

16. Mons coagulátus, mons \textbf{pin}guis: \ast\  ut quid suspicámini montes coa\textit{gu}\textbf{lá}tos?\

17. Mons, in quo beneplácitum est Deo habitáre in \textbf{e}o: \ast\  étenim Dóminus habitábit \textit{in} \textbf{fi}nem.\

18. Currus Dei decem míllibus múltiplex, míllia læ\textbf{tán}tium: \ast\  Dóminus in eis in Sina \textit{in} \textbf{sanc}to.\

19. Ascendísti in altum, cepísti captivi\textbf{tá}tem: \ast\  accepísti dona in \textit{ho}\textbf{mí}\textbf{ni}bus.\

20. Etenim non cre\textbf{dén}tes, \ast\  inhabitáre Dómi\textit{num} \textbf{De}um.\

21. Benedíctus Dóminus die quo\textbf{tí}die: \ast\  prósperum iter fáciet nobis Deus salutárium \textit{nos}\textbf{tró}rum.\

22. Deus noster, Deus salvos faci\textbf{én}di: \ast\  et Dómini Dómini éxi\textit{tus} \textbf{mor}tis.\

23. Verúmtamen Deus confrínget cápita inimicórum su\textbf{ó}rum: \ast\  vérticem capílli perambulántium in delíc\textit{tis} \textbf{su}is.\

24. Dixit Dóminus: Ex Basan con\textbf{vér}tam, \ast\  convértam in profún\textit{dum} \textbf{ma}ris:\

25. Ut intingátur pes tuus in \textbf{sán}guine: \ast\  lingua canum tuórum ex inimícis, \textit{ab} \textbf{ip}so.\

26. Vidérunt ingréssus tuos, \textbf{De}us: \ast\  ingréssus Dei mei: regis mei qui est \textit{in} \textbf{sanc}to.\

27. Prævenérunt príncipes conjúncti psal\textbf{lén}tibus: \ast\  in médio juvenculárum tympanis\textit{tri}\textbf{á}rum.\

28. In ecclésiis benedícite Deo \textbf{Dó}mino, \ast\  de fónti\textit{bus} \textbf{Is}\textbf{ra}ël.\

29. Ibi Bénjamin adole\textbf{scén}tulus: \ast\  in mentis \textit{ex}\textbf{cés}su.\

30. Príncipes Juda, duces e\textbf{ó}rum: \ast\  príncipes Zábulon, prínci\textit{pes} \textbf{Néph}\textbf{ta}li.\

31. Manda, Deus, virtúti \textbf{tu}æ: \ast\  confírma hoc, Deus, quod operátus es \textit{in} \textbf{no}bis.\

32. A templo tuo in Je\textbf{rú}salem, \ast\  tibi ófferent re\textit{ges} \textbf{mú}\textbf{ne}ra.\

33. Increpa feras arúndinis, congregátio taurórum in vaccis popu\textbf{ló}rum: \ast\  ut exclúdant eos, qui probáti sunt \textit{ar}\textbf{gén}to.\

34. Díssipa Gentes, quæ bella volunt: vénient legáti ex Æ\textbf{gýp}to: \ast\  Æthiópia prævéniet manus e\textit{jus} \textbf{De}o.\

35. Regna terræ, cantáte \textbf{De}o: \ast\  psálli\textit{te} \textbf{Dó}\textbf{mi}no.\

36. Psállite Deo, qui ascéndit super cælum \textbf{cæ}li, \ast\  ad O\textit{ri}\textbf{én}tem.\

37. Ecce dabit voci suæ vocem virtútis, date glóriam Deo super \textbf{Is}raël, \ast\  magnificéntia ejus, et virtus ejus \textit{in} \textbf{nú}\textbf{bi}bus.\

38. Mirábilis Deus in sanctis suis, Deus Israël ipse dabit virtútem, et fortitúdinem plebi \textbf{su}æ, \ast\  benedíc\textit{tus} \textbf{De}us.\

