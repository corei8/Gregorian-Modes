2. Posuérunt morticína servórum tuórum, escas vola\textbf{tí}libus \textbf{cæ}li: \ast\  carnes sanctórum tuórum \textit{bés}\textit{ti}\textit{is} \textbf{ter}ræ.\

3. Effudérunt sánguinem eórum tamquam aquam in circúi\textbf{tu} Je\textbf{rú}\textbf{sa}lem: \ast\  et non erat \textit{qui} \textit{se}\textit{pe}\textbf{lí}ret.\

4. Facti sumus oppróbrium vi\textbf{cí}nis \textbf{nos}tris: \ast\  subsannátio et illúsio his, qui in cir\textit{cú}\textit{i}\textit{tu} \textbf{nos}tro sunt.\

5. Usquequo, Dómine, ira\textbf{scé}ris in \textbf{fi}nem: \ast\  accendétur velut i\textit{gnis} \textit{ze}\textit{lus} \textbf{tu}us?\

6. Effúnde iram tuam in Gentes, quæ te \textbf{non} no\textbf{vé}runt: \ast\  et in regna quæ nomen tuum non \textit{in}\textit{vo}\textit{ca}\textbf{vé}runt:\

7. Quia come\textbf{dé}runt \textbf{Ja}cob: \ast\  et locum ejus \textit{de}\textit{so}\textit{la}\textbf{vé}runt.\

8. Ne memíneris iniquitátum nostrárum antiquárum, cito antícipent nos miseri\textbf{cór}diæ \textbf{tu}æ: \ast\  quia páuperes fac\textit{ti} \textit{su}\textit{mus} \textbf{ni}mis.\

9. Adjuva nos, Deus, salutáris noster: et propter glóriam nóminis tui, Dómine, \textbf{lí}be\textbf{ra} nos: \ast\  et propítius esto peccátis nostris, prop\textit{ter} \textit{no}\textit{men} \textbf{tu}um:\

10. Ne forte dicant in Géntibus: Ubi est \textbf{De}us e\textbf{ó}rum? \ast\  et innotéscat in natiónibus coram \textit{ó}\textit{cu}\textit{lis} \textbf{nos}tris.\

11. Ultio sánguinis servórum tuórum, \textbf{qui} ef\textbf{fú}\textbf{sus} est: \ast\  intróeat in conspéctu tuo gémitus \textit{com}\textit{pe}\textit{di}\textbf{tó}rum.\

12. Secúndum magnitúdinem \textbf{brá}chii \textbf{tu}i, \ast\  pósside fílios mor\textit{ti}\textit{fi}\textit{ca}\textbf{tó}rum.\

13. Et redde vicínis nostris séptuplum in \textbf{si}nu e\textbf{ó}rum: \ast\  impropérium ipsórum, quod exprobravé\textit{runt} \textit{ti}\textit{bi}, \textbf{Dó}mine.\

14. Nos autem pópulus tuus, et oves \textbf{pás}cuæ \textbf{tu}æ, \ast\  confitébimur \textit{ti}\textit{bi} \textit{in} \textbf{sǽ}culum.\

15. In generatiónem et gene\textbf{ra}ti\textbf{ó}nem \ast\  annuntiábi\textit{mus} \textit{lau}\textit{dem} \textbf{tu}am.\

