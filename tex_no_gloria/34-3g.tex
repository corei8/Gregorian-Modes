2. Apprehénde \textbf{ar}ma et \textbf{scu}tum: \ast\  et exsúrge in adjutó\textit{ri}\textit{um} \textbf{mi}hi.\

3. Effúnde frámeam, et conclúde advérsus eos, qui \textbf{per}se\textbf{quún}\textbf{tur} me: \ast\  dic ánimæ meæ: Salus \textit{tu}\textit{a} \textbf{e}go sum.\

4. Confundántur et re\textbf{ve}re\textbf{án}tur, \ast\  quæréntes á\textit{ni}\textit{mam} \textbf{me}am.\

5. Avertántur retrórsum, et \textbf{con}fun\textbf{dán}tur \ast\  cogitántes \textit{mi}\textit{hi} \textbf{ma}la.\

6. Fiant tamquam pulvis ante \textbf{fá}ciem \textbf{ven}ti: \ast\  et Angelus Dómini co\textit{árc}\textit{tans} \textbf{e}os.\

7. Fiat via illórum téne\textbf{bræ} et \textbf{lú}\textbf{bri}cum: \ast\  et Angelus Dómini pér\textit{se}\textit{quens} \textbf{e}os.\

8. Quóniam gratis abscondérunt mihi intéritum \textbf{lá}quei \textbf{su}i: \ast\  supervácue exprobravérunt á\textit{ni}\textit{mam} \textbf{me}am.\

9. Véniat illi láqueus, quem ignórat: \dag\  et cáptio, quam abscóndit, appre\textbf{hén}dat \textbf{e}um: \ast\  et in láqueum ca\textit{dat} \textit{in} \textbf{ip}sum.\

10. Anima autem mea exsul\textbf{tá}bit in \textbf{Dó}\textbf{mi}no: \ast\  et delectábitur super salu\textit{tá}\textit{ri} \textbf{su}o.\

11. Omnia ossa \textbf{me}a \textbf{di}cent: \ast\  Dómine, quis sí\textit{mi}\textit{lis} \textbf{ti}bi?\

12. Erípiens ínopem de manu forti\textbf{ó}rum \textbf{e}jus: \ast\  egénum et páuperem a diripién\textit{ti}\textit{bus} \textbf{e}um.\

13. Surgéntes \textbf{tes}tes in\textbf{í}qui, \ast\  quæ ignorábam in\textit{ter}\textit{ro}\textbf{gá}bant me.\

14. Retribuébant mihi \textbf{ma}la pro \textbf{bo}nis: \ast\  sterilitátem á\textit{ni}\textit{mæ} \textbf{me}æ.\

15. Ego autem cum mihi mo\textbf{lés}ti \textbf{es}sent, \ast\  indué\textit{bar} \textit{ci}\textbf{lí}cio.\

16. Humiliábam in jejúnio \textbf{á}nimam \textbf{me}am: \ast\  et orátio mea in sinu meo \textit{con}\textit{ver}\textbf{té}tur.\

17. Quasi próximum, et quasi fratrem nostrum, sic \textbf{com}pla\textbf{cé}bam: \ast\  quasi lugens et contristátus, sic hu\textit{mi}\textit{li}\textbf{á}bar.\

18. Et advérsum me lætáti sunt, et \textbf{con}ve\textbf{né}runt: \ast\  congregáta sunt super me flagélla, et \textit{i}\textit{gno}\textbf{rá}vi.\

19. Dissipáti sunt, nec compúncti, \dag\  tentavérunt me, subsannavérunt me subsan\textbf{na}ti\textbf{ó}ne: \ast\  frenduérunt super me dén\textit{ti}\textit{bus} \textbf{su}is.\

20. Dómine, \textbf{quan}do re\textbf{spí}\textbf{ci}es? \ast\  restítue ánimam meam a malignitáte eórum, a leónibus ú\textit{ni}\textit{cam} \textbf{me}am.\

21. Confitébor tibi in ec\textbf{clé}sia \textbf{ma}gna, \ast\  in pópulo gra\textit{vi} \textit{lau}\textbf{dá}bo te.\

22. Non supergáudeant mihi qui adversántur \textbf{mi}hi in\textbf{í}que: \ast\  qui odérunt me gratis et án\textit{nu}\textit{unt} \textbf{ó}culis.\

23. Quóniam mihi quidem pacífice \textbf{lo}que\textbf{bán}tur: \ast\  et in iracúndia terræ loquéntes, dolos \textit{co}\textit{gi}\textbf{tá}bant.\

24. Et dilatavérunt super \textbf{me} os \textbf{su}um: \ast\  dixérunt: Euge, euge, vidérunt ó\textit{cu}\textit{li} \textbf{nos}tri.\

25. Vidísti, Dómi\textbf{ne}, ne \textbf{sí}\textbf{le}as: \ast\  Dómine, ne di\textit{scé}\textit{das} \textbf{a} me.\

26. Exsúrge et inténde ju\textbf{dí}cio \textbf{me}o: \ast\  Deus meus, et Dóminus meus in \textit{cau}\textit{sam} \textbf{me}am.\

27. Júdica me secúndum justítiam tuam, Dómine, \textbf{De}us \textbf{me}us, \ast\  et non supergáu\textit{de}\textit{ant} \textbf{mi}hi.\

28. Non dicant in córdibus suis: \dag\  Euge, euge, \textbf{á}nimæ \textbf{nos}træ: \ast\  nec dicant: Devorá\textit{vi}\textit{mus} \textbf{e}um.\

29. Erubéscant et revere\textbf{án}tur \textbf{si}mul, \ast\  qui gratulántur \textit{ma}\textit{lis} \textbf{me}is.\

30. Induántur confusióne et \textbf{re}ve\textbf{rén}\textbf{ti}a \ast\  qui magna lo\textit{quún}\textit{tur} \textbf{su}per me.\

31. Exsúltent et læténtur qui volunt jus\textbf{tí}tiam \textbf{me}am: \ast\  et dicant semper: Magnificétur Dóminus qui volunt pacem \textit{ser}\textit{vi} \textbf{e}jus.\

32. Et lingua mea meditábitur jus\textbf{tí}tiam \textbf{tu}am, \ast\  tota die \textit{lau}\textit{dem} \textbf{tu}am.\

