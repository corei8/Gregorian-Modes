2. Quóniam dolóse egit in \textit{con}\textit{spéc}\textit{tu} \textbf{e}jus: \ast\  ut inveniátur iníquitas ejus \textit{ad} \textbf{ó}dium.\

3. Verba oris ejus iní\textit{qui}\textit{tas}, \textit{et} \textbf{do}lus: \ast\  nóluit intellígere ut be\textit{ne} \textbf{á}geret.\

4. Iniquitátem meditátus est in \textit{cu}\textit{bí}\textit{li} \textbf{su}o: \ast\  ástitit omni viæ non bonæ, malítiam autem non \textit{o}\textbf{dí}vit.\

5. Dómine, in cælo miseri\textit{cór}\textit{di}\textit{a} \textbf{tu}a: \ast\  et véritas tua usque \textit{ad} \textbf{nu}bes.\

6. Justítia tua sic\textit{ut} \textit{mon}\textit{tes} \textbf{De}i: \ast\  judícia tua abýs\textit{sus} \textbf{mul}ta.\

7. Hómines, et juménta \textit{sal}\textit{vá}\textit{bis}, \textbf{Dó}mine: \ast\  quemádmodum multiplicásti misericórdiam tu\textit{am}, \textbf{De}us,\

8. Fíli\textit{i} \textit{au}\textit{tem} \textbf{hó}minum, \ast\  in tégmine alárum tuárum \textit{spe}\textbf{rá}bunt.\

9. Inebriabúntur ab ubertá\textit{te} \textit{do}\textit{mus} \textbf{tu}æ: \ast\  et torrénte voluptátis tuæ potá\textit{bis} \textbf{e}os.\

10. Quóniam apud \textit{te} \textit{est} \textit{fons} \textbf{vi}tæ: \ast\  et in lúmine tuo vidébi\textit{mus} \textbf{lu}men.\

11. Præténde misericórdiam tuam \textit{sci}\textit{én}\textit{ti}\textbf{bus} te, \ast\  et justítiam tuam his, qui recto \textit{sunt} \textbf{cor}de.\

12. Non véniat mi\textit{hi} \textit{pes} \textit{su}\textbf{pér}biæ: \ast\  et manus peccatóris non mó\textit{ve}\textbf{at} me.\

13. Ibi cecidérunt qui operántur \textit{in}\textit{i}\textit{qui}\textbf{tá}tem: \ast\  expúlsi sunt, nec potué\textit{runt} \textbf{sta}re.\

