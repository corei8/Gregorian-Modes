2. Apprehénde \textit{ar}\textit{ma} \textit{et} \textbf{scu}tum: \ast\  et exsúrge in adjutóri\textit{um} \textbf{mi}hi.\

3. Effúnde frámeam, et conclúde advérsus eos, \textit{qui} \textit{per}\textit{se}\textbf{quún}tur me: \ast\  dic ánimæ meæ: Salus tu\textit{a} \textbf{e}go sum.\

4. Confundántur et \textit{re}\textit{ve}\textit{re}\textbf{án}tur, \ast\  quæréntes áni\textit{mam} \textbf{me}am.\

5. Avertántur retrórsum, \textit{et} \textit{con}\textit{fun}\textbf{dán}tur \ast\  cogitántes mi\textit{hi} \textbf{ma}la.\

6. Fiant tamquam pulvis ante \textit{fá}\textit{ci}\textit{em} \textbf{ven}ti: \ast\  et Angelus Dómini coárc\textit{tans} \textbf{e}os.\

7. Fiat via illórum té\textit{ne}\textit{bræ} \textit{et} \textbf{lú}bricum: \ast\  et Angelus Dómini pérse\textit{quens} \textbf{e}os.\

8. Quóniam gratis abscondérunt mihi intéritum \textit{lá}\textit{que}\textit{i} \textbf{su}i: \ast\  supervácue exprobravérunt áni\textit{mam} \textbf{me}am.\

9. Véniat illi láqueus, quem ignórat: \dag\  et cáptio, quam abscóndit, ap\textit{pre}\textit{hén}\textit{dat} \textbf{e}um: \ast\  et in láqueum cadat \textit{in} \textbf{ip}sum.\

10. Anima autem mea exsul\textit{tá}\textit{bit} \textit{in} \textbf{Dó}mino: \ast\  et delectábitur super salutá\textit{ri} \textbf{su}o.\

11. Omnia os\textit{sa} \textit{me}\textit{a} \textbf{di}cent: \ast\  Dómine, quis sími\textit{lis} \textbf{ti}bi?\

12. Erípiens ínopem de manu for\textit{ti}\textit{ó}\textit{rum} \textbf{e}jus: \ast\  egénum et páuperem a diripiénti\textit{bus} \textbf{e}um.\

13. Surgéntes \textit{tes}\textit{tes} \textit{in}\textbf{í}qui, \ast\  quæ ignorábam inter\textit{ro}\textbf{gá}bant me.\

14. Retribuébant mihi \textit{ma}\textit{la} \textit{pro} \textbf{bo}nis: \ast\  sterilitátem áni\textit{mæ} \textbf{me}æ.\

15. Ego autem cum mihi \textit{mo}\textit{lés}\textit{ti} \textbf{es}sent, \ast\  induébar \textit{ci}\textbf{lí}cio.\

16. Humiliábam in jejúnio \textit{á}\textit{ni}\textit{mam} \textbf{me}am: \ast\  et orátio mea in sinu meo con\textit{ver}\textbf{té}tur.\

17. Quasi próximum, et quasi fratrem nostrum, \textit{sic} \textit{com}\textit{pla}\textbf{cé}bam: \ast\  quasi lugens et contristátus, sic humi\textit{li}\textbf{á}bar.\

18. Et advérsum me lætáti sunt, \textit{et} \textit{con}\textit{ve}\textbf{né}runt: \ast\  congregáta sunt super me flagélla, et i\textit{gno}\textbf{rá}vi.\

19. Dissipáti sunt, nec compúncti, \dag\  tentavérunt me, subsannavérunt me sub\textit{san}\textit{na}\textit{ti}\textbf{ó}ne: \ast\  frenduérunt super me dénti\textit{bus} \textbf{su}is.\

20. Dómine, \textit{quan}\textit{do} \textit{re}\textbf{spí}cies? \ast\  restítue ánimam meam a malignitáte eórum, a leónibus úni\textit{cam} \textbf{me}am.\

21. Confitébor tibi in ec\textit{clé}\textit{si}\textit{a} \textbf{ma}gna, \ast\  in pópulo gravi \textit{lau}\textbf{dá}bo te.\

22. Non supergáudeant mihi qui adversántur \textit{mi}\textit{hi} \textit{in}\textbf{í}que: \ast\  qui odérunt me gratis et ánnu\textit{unt} \textbf{ó}culis.\

23. Quóniam mihi quidem pacífi\textit{ce} \textit{lo}\textit{que}\textbf{bán}tur: \ast\  et in iracúndia terræ loquéntes, dolos co\textit{gi}\textbf{tá}bant.\

24. Et dilatavérunt su\textit{per} \textit{me} \textit{os} \textbf{su}um: \ast\  dixérunt: Euge, euge, vidérunt ócu\textit{li} \textbf{nos}tri.\

25. Vidísti, Dó\textit{mi}\textit{ne}, \textit{ne} \textbf{sí}leas: \ast\  Dómine, ne discé\textit{das} \textbf{a} me.\

26. Exsúrge et inténde ju\textit{dí}\textit{ci}\textit{o} \textbf{me}o: \ast\  Deus meus, et Dóminus meus in cau\textit{sam} \textbf{me}am.\

27. Júdica me secúndum justítiam tuam, Dómi\textit{ne}, \textit{De}\textit{us} \textbf{me}us, \ast\  et non supergáude\textit{ant} \textbf{mi}hi.\

28. Non dicant in córdibus suis: \dag\  Euge, euge, \textit{á}\textit{ni}\textit{mæ} \textbf{nos}træ: \ast\  nec dicant: Devorávi\textit{mus} \textbf{e}um.\

29. Erubéscant et reve\textit{re}\textit{án}\textit{tur} \textbf{si}mul, \ast\  qui gratulántur ma\textit{lis} \textbf{me}is.\

30. Induántur confusióne \textit{et} \textit{re}\textit{ve}\textbf{rén}tia \ast\  qui magna loquún\textit{tur} \textbf{su}per me.\

31. Exsúltent et læténtur qui volunt jus\textit{tí}\textit{ti}\textit{am} \textbf{me}am: \ast\  et dicant semper: Magnificétur Dóminus qui volunt pacem ser\textit{vi} \textbf{e}jus.\

32. Et lingua mea meditábitur jus\textit{tí}\textit{ti}\textit{am} \textbf{tu}am, \ast\  tota die lau\textit{dem} \textbf{tu}am.\

