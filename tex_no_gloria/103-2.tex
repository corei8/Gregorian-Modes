2. Confessiónem, et decórem indu\textbf{ís}ti: \ast\  amíctus lúmine sicut ves\textit{ti}\textbf{mén}to.\

3. Exténdens cælum sicut \textbf{pel}lem: \ast\  qui tegis aquis superió\textit{ra} \textbf{e}jus.\

4. Qui ponis nubem ascénsum \textbf{tu}um: \ast\  qui ámbulas super pennas \textit{ven}\textbf{tó}rum.\

5. Qui facis ángelos tuos, \textbf{spí}ritus: \ast\  et minístros tuos ignem \textit{u}\textbf{rén}tem.\

6. Qui fundásti terram super stabilitátem \textbf{su}am: \ast\  non inclinábitur in sǽcu\textit{lum} \textbf{sǽ}culi.\

7. Abýssus, sicut vestiméntum, amíctus \textbf{e}jus: \ast\  super montes sta\textit{bunt} \textbf{a}quæ.\

8. Ab increpatióne tua \textbf{fú}gient: \ast\  a voce tonítrui tui for\textit{mi}\textbf{dá}bunt.\

9. Ascéndunt montes: et descéndunt \textbf{cam}pi \ast\  in locum, quem fundás\textit{ti} \textbf{e}is.\

10. Términum posuísti, quem non transgredi\textbf{én}tur: \ast\  neque converténtur operí\textit{re} \textbf{ter}ram.\

11. Qui emíttis fontes in con\textbf{vál}libus: \ast\  inter médium móntium pertransí\textit{bunt} \textbf{a}quæ.\

12. Potábunt omnes béstiæ \textbf{a}gri: \ast\  exspectábunt ónagri in si\textit{ti} \textbf{su}a.\

13. Super ea vólucres cæli habi\textbf{tá}bunt: \ast\  de médio petrárum da\textit{bunt} \textbf{vo}ces.\

14. Rigans montes de superióribus \textbf{su}is: \ast\  de fructu óperum tuórum satiábi\textit{tur} \textbf{ter}ra:\

15. Prodúcens fœnum ju\textbf{mén}tis: \ast\  et herbam servitú\textit{ti} \textbf{hó}minum:\

16. Ut edúcas panem de \textbf{ter}ra: \ast\  et vinum lætíficet \textit{cor} \textbf{hó}minis:\

17. Ut exhílaret fáciem in \textbf{ó}leo: \ast\  et panis cor hóminis \textit{con}\textbf{fír}met.\

18. Saturabúntur ligna campi, et cedri Líbani, quas plan\textbf{tá}vit: \ast\  illic pásseres nidi\textit{fi}\textbf{cá}bunt.\

19. Heródii domus dux est e\textbf{ó}rum: \ast\  montes excélsi cervis: petra refúgium he\textit{ri}\textbf{ná}ciis.\

20. Fecit lunam in \textbf{tém}pora: \ast\  sol cognóvit occá\textit{sum} \textbf{su}um.\

21. Posuísti ténebras, et facta \textbf{est} nox: \ast\  in ipsa pertransíbunt omnes bésti\textit{æ} \textbf{sil}væ.\

22. Cátuli leónum rugiéntes, ut \textbf{rá}piant: \ast\  et quærant a Deo es\textit{cam} \textbf{si}bi.\

23. Ortus est sol, et congre\textbf{gá}ti sunt: \ast\  et in cubílibus suis collo\textit{ca}\textbf{bún}tur.\

24. Exíbit homo ad opus \textbf{su}um: \ast\  et ad operatiónem suam usque \textit{ad} \textbf{vés}perum.\

25. Quam magnificáta sunt ópera tua, \textbf{Dó}mine! \ast\  ómnia in sapiéntia fecísti: impléta est terra possessió\textit{ne} \textbf{tu}a.\

26. Hoc mare magnum, et spatiósum \textbf{má}nibus: \ast\  illic reptília, quorum non \textit{est} \textbf{nú}merus.\

27. Animália pusílla cum \textbf{ma}gnis: \ast\  illic naves per\textit{trans}\textbf{í}bunt.\

28. Draco iste, quem formásti ad illudéndum \textbf{e}i: \ast\  ómnia a te exspéctant ut des illis escam \textit{in} \textbf{tém}pore.\

29. Dante te illis, \textbf{cól}ligent: \ast\  aperiénte te manum tuam, ómnia implebúntur bo\textit{ni}\textbf{tá}te.\

30. Averténte autem te fáciem, turba\textbf{bún}tur: \ast\  áuferes spíritum eórum, et defícient, et in púlverem suum re\textit{ver}\textbf{tén}tur.\

31. Emíttes spíritum tuum, et crea\textbf{bún}tur: \ast\  et renovábis fáci\textit{em} \textbf{ter}ræ.\

32. Sit glória Dómini in \textbf{sǽ}culum: \ast\  lætábitur Dóminus in opéri\textit{bus} \textbf{su}is:\

33. Qui réspicit terram, et facit eam \textbf{tré}mere: \ast\  qui tangit montes, \textit{et} \textbf{fú}migant.\

34. Cantábo Dómino in vita \textbf{me}a: \ast\  psallam Deo meo, quám\textit{di}\textbf{u} sum.\

35. Jucúndum sit ei elóquium \textbf{me}um: \ast\  ego vero delectábor \textit{in} \textbf{Dó}mino.\

36. Defíciant peccatóres a terra, et iníqui ita ut \textbf{non} sint: \ast\  bénedic, ánima me\textit{a}, \textbf{Dó}mino.\

