% !TEX program = lualatex
\documentclass{amsbook}

\author{}

\usepackage{geometry}
\geometry{paperwidth=8.5in, paperheight=11in, left=1.0in, right=1.0in, top=1.0in, bottom=1.0in,}

% GREGORIO
\usepackage[autocompile]{gregoriotex}
\gresetlinecolor{gregoriocolor}

\thispagestyle{empty}

\begin{document}
    \gregorioscore{103-4-antiquo-alt-A.gabc}
    2. Confessiónem, et decórem 	extit{ind}	extit{u}	extbf{ís}ti: st\  amíctus lúmine sic	extit{ut} 	extit{ves}	extit{ti}	extbf{mén}to.\

3. Exténdens cælum 	extit{sic}	extit{ut} 	extbf{pel}lem: st\  qui tegis aquis supe	extit{ri}	extit{ó}	extit{ra} 	extbf{e}jus.\

4. Qui ponis nubem a	extit{scén}	extit{sum} 	extbf{tu}um: st\  qui ámbulas super 	extit{pen}	extit{nas} 	extit{ven}	extbf{tó}rum.\

5. Qui facis ángelos 	extit{tu}	extit{os}, 	extbf{spí}ritus: st\  et minístros tuos 	extit{i}	extit{gnem} 	extit{u}	extbf{rén}tem.\

6. Qui fundásti terram super stabili	extit{tá}	extit{tem} 	extbf{su}am: st\  non inclinábitur in 	extit{sǽ}	extit{cu}	extit{lum} 	extbf{sǽ}culi.\

7. Abýssus, sicut vestiméntum, a	extit{míc}	extit{tus} 	extbf{e}jus: st\  super mon	extit{tes} 	extit{sta}	extit{bunt} 	extbf{a}quæ.\

8. Ab increpatióne 	extit{tu}	extit{a} 	extbf{fú}gient: st\  a voce tonítrui tu	extit{i} 	extit{for}	extit{mi}	extbf{dá}bunt.\

9. Ascéndunt montes: et de	extit{scén}	extit{dunt} 	extbf{cam}pi st\  in locum, quem 	extit{fun}	extit{dás}	extit{ti} 	extbf{e}is.\

10. Términum posuísti, quem non trans	extit{gre}	extit{di}	extbf{én}tur: st\  neque converténtur o	extit{pe}	extit{rí}	extit{re} 	extbf{ter}ram.\

11. Qui emíttis fontes 	extit{in} 	extit{con}	extbf{vál}libus: st\  inter médium móntium per	extit{trans}	extit{í}	extit{bunt} 	extbf{a}quæ.\

12. Potábunt omnes bés	extit{ti}	extit{æ} 	extbf{a}gri: st\  exspectábunt ónagri 	extit{in} 	extit{si}	extit{ti} 	extbf{su}a.\

13. Super ea vólucres cæli 	extit{ha}	extit{bi}	extbf{tá}bunt: st\  de médio petrá	extit{rum} 	extit{da}	extit{bunt} 	extbf{vo}ces.\

14. Rigans montes de superió	extit{ri}	extit{bus} 	extbf{su}is: st\  de fructu óperum tuórum sati	extit{á}	extit{bi}	extit{tur} 	extbf{ter}ra:\

15. Prodúcens fœ	extit{num} 	extit{ju}	extbf{mén}tis: st\  et herbam ser	extit{vi}	extit{tú}	extit{ti} 	extbf{hó}minum:\

16. Ut edúcas pa	extit{nem} 	extit{de} 	extbf{ter}ra: st\  et vinum lætí	extit{fi}	extit{cet} 	extit{cor} 	extbf{hó}minis:\

17. Ut exhílaret fáci	extit{em} 	extit{in} 	extbf{ó}leo: st\  et panis cor hó	extit{mi}	extit{nis} 	extit{con}	extbf{fír}met.\

18. Saturabúntur ligna campi, et cedri Líbani, 	extit{quas} 	extit{plan}	extbf{tá}vit: st\  illic pásseres 	extit{ni}	extit{di}	extit{fi}	extbf{cá}bunt.\

19. Heródii domus dux 	extit{est} 	extit{e}	extbf{ó}rum: st\  montes excélsi cervis: petra refúgi	extit{um} 	extit{he}	extit{ri}	extbf{ná}ciis.\

20. Fecit lu	extit{nam} 	extit{in} 	extbf{tém}pora: st\  sol cognóvit 	extit{oc}	extit{cá}	extit{sum} 	extbf{su}um.\

21. Posuísti ténebras, et 	extit{fac}	extit{ta} 	extbf{est} nox: st\  in ipsa pertransíbunt omnes 	extit{bés}	extit{ti}	extit{æ} 	extbf{sil}væ.\

22. Cátuli leónum rugién	extit{tes}, 	extit{ut} 	extbf{rá}piant: st\  et quærant a De	extit{o} 	extit{es}	extit{cam} 	extbf{si}bi.\

23. Ortus est sol, et 	extit{con}	extit{gre}	extbf{gá}ti sunt: st\  et in cubílibus suis 	extit{col}	extit{lo}	extit{ca}	extbf{bún}tur.\

24. Exíbit homo ad 	extit{o}	extit{pus} 	extbf{su}um: st\  et ad operatiónem suam 	extit{us}	extit{que} 	extit{ad} 	extbf{vés}perum.\

25. Quam magnificáta sunt ópera 	extit{tu}	extit{a}, 	extbf{Dó}mine! st\  ómnia in sapiéntia fecísti: impléta est terra posses	extit{si}	extit{ó}	extit{ne} 	extbf{tu}a.\

26. Hoc mare magnum, et spati	extit{ó}	extit{sum} 	extbf{má}nibus: st\  illic reptília, quo	extit{rum} 	extit{non} 	extit{est} 	extbf{nú}merus.\

27. Animália pusíl	extit{la} 	extit{cum} 	extbf{ma}gnis: st\  illic na	extit{ves} 	extit{per}	extit{trans}	extbf{í}bunt.\

28. Draco iste, quem formásti ad illu	extit{dén}	extit{dum} 	extbf{e}i: st\  ómnia a te exspéctant ut des illis 	extit{es}	extit{cam} 	extit{in} 	extbf{tém}pore.\

29. Dante te 	extit{il}	extit{lis}, 	extbf{cól}ligent: st\  aperiénte te manum tuam, ómnia implebún	extit{tur} 	extit{bo}	extit{ni}	extbf{tá}te.\

30. Averténte autem te fáciem, 	extit{tur}	extit{ba}	extbf{bún}tur: st\  áuferes spíritum eórum, et defícient, et in púlverem su	extit{um} 	extit{re}	extit{ver}	extbf{tén}tur.\

31. Emíttes spíritum tuum, et 	extit{cre}	extit{a}	extbf{bún}tur: st\  et renovábis 	extit{fá}	extit{ci}	extit{em} 	extbf{ter}ræ.\

32. Sit glória Dómi	extit{ni} 	extit{in} 	extbf{sǽ}culum: st\  lætábitur Dóminus in o	extit{pé}	extit{ri}	extit{bus} 	extbf{su}is:\

33. Qui réspicit terram, et facit 	extit{e}	extit{am} 	extbf{tré}mere: st\  qui tangit 	extit{mon}	extit{tes}, 	extit{et} 	extbf{fú}migant.\

34. Cantábo Dómino in 	extit{vi}	extit{ta} 	extbf{me}a: st\  psallam Deo me	extit{o}, 	extit{quám}	extit{di}	extbf{u} sum.\

35. Jucúndum sit ei eló	extit{qui}	extit{um} 	extbf{me}um: st\  ego vero delec	extit{tá}	extit{bor} 	extit{in} 	extbf{Dó}mino.\

36. Defíciant peccatóres a terra, et iníqui i	extit{ta} 	extit{ut} 	extbf{non} sint: st\  bénedic, áni	extit{ma} 	extit{me}	extit{a}, 	extbf{Dó}mino.\

37. Glória Pa	extit{tri}, 	extit{et} 	extbf{Fí}lio, st\  et Spi	extit{rí}	extit{tu}	extit{i} 	extbf{Sanc}to.\

38. Sicut erat in princípio, et 	extit{nunc}, 	extit{et} 	extbf{sem}per, st\  et in sǽcula sæ	extit{cu}	extit{ló}	extit{rum}. 	extbf{A}men.\


\end{document}
